{\rtf1\ansi\ansicpg1252\cocoartf2511
\cocoatextscaling0\cocoaplatform0{\fonttbl\f0\fswiss\fcharset0 Helvetica;}
{\colortbl;\red255\green255\blue255;}
{\*\expandedcolortbl;;}
\margl1440\margr1440\vieww10800\viewh8400\viewkind0
\pard\tx720\tx1440\tx2160\tx2880\tx3600\tx4320\tx5040\tx5760\tx6480\tx7200\tx7920\tx8640\pardirnatural\partightenfactor0

\f0\fs24 \cf0 %%%%%%%%%%%%%%%%%%%%%%%%%%%%%%%%%%%%%%%%%%%%%%%%%%%%%%%%%%%%%%%%%%%%%%%%%%%%%%%%%%%%\
% Do not alter this block (unless you're familiar with LaTeX\
\\documentclass\{article\}\
\\usepackage[margin=1in]\{geometry\} \
\\usepackage\{amsmath,amsthm,amssymb,amsfonts, fancyhdr, color, comment, graphicx, environ\}\
\\usepackage\{xcolor\}\
\\usepackage\{mdframed\}\
\\usepackage[shortlabels]\{enumitem\}\
\\usepackage\{indentfirst\}\
\\usepackage\{hyperref\}\
\\usepackage\{soul\}\
\\usepackage\{gensymb\}\
\\usepackage\{tikz\}\
\\hypersetup\{\
    colorlinks=true,\
    linkcolor=blue,\
    filecolor=magenta,      \
    urlcolor=blue,\
\}\
\
\
\\pagestyle\{fancy\}\
\
\
\\newenvironment\{problem\}[3][Problem]\
    \{ \\begin\{mdframed\}[backgroundcolor=gray!20] \\textbf\{#1 #2\} \\textit\{worth #3 points\} \\\\\}\
    \{  \\end\{mdframed\}\}\
\
% Define solution environment\
\\newenvironment\{solution\}\
    \{\\textit\{Solution:\}\}\
    \{\}\
\
\\renewcommand\{\\qed\}\{\\quad\\qedsymbol\}\
\
% prevent line break in inline mode\
\\binoppenalty=\\maxdimen\
\\relpenalty=\\maxdimen\
\
%%%%%%%%%%%%%%%%%%%%%%%%%%%%%%%%%%%%%%%%%%%%%\
%Fill in the appropriate information below\
\\lhead\{Will Kanter\}\
\\rhead\{CSCI 2820\} \
\\chead\{\\textbf\{Homework 3 Due: 7 February 2020 11:59 PM\}\}\
%%%%%%%%%%%%%%%%%%%%%%%%%%%%%%%%%%%%%%%%%%%%%\
\
\\begin\{document\}\
\
\\begin\{problem\}\{3.4\}\{4\}\
\\textit\{Norm identities\}. Verify that the following identities hold for any two vectors $a$ and $b$ of the same size.\
\\begin\{enumerate\}[(a)]\
    \\item $(a+b)^T(a-b) = \\lvert\\lvert a \\rvert\\rvert^2 - \\lvert\\lvert b \\rvert\\rvert^2$\
    \\item $\\lvert\\lvert a + b \\rvert\\rvert^2 + \\lvert\\lvert a - b \\rvert\\rvert^2 = 2(\\lvert\\lvert a \\rvert\\rvert^2 + \\lvert\\lvert b \\rvert\\rvert^2)$ This is called the \\textit\{parallelogram law\}.\
\\end\{enumerate\}\
\\end\{problem\}\
\\begin\{solution\}\
\\begin\{enumerate\}[(a)]\
    \\item $(a+b)^T(a-b)=\\lvert\\lvert a \\rvert\\rvert^2 - \\lvert\\lvert b \\rvert\\rvert^2$\\\\\
    $(a+b)^T(a-b) = a^Ta - a^Tb + a^Tb - b^Tb$\\\\\
    $(a+b)^T(a-b) = a^Ta - b^Tb$\\\\\
    $\\lvert\\lvert a \\rvert\\rvert = \\sqrt\{a^Ta\}$\\\\\
    $\\lvert\\lvert b \\rvert\\rvert = \\sqrt\{b^Tb\}$\\\\\
    $\\lvert\\lvert a \\rvert\\rvert^2 = a^Ta$\\\\\
    $\\lvert\\lvert b \\rvert\\rvert^2 = b^Tb$\\\\\
    $\\therefore(a + b)^T(a - b) = a^Ta - b^Tb = \\lvert\\lvert a \\rvert\\rvert^2 - \\lvert\\lvert b \\rvert\\rvert^2$\\\\\
    \\item $\\lvert\\lvert a + b \\rvert\\rvert^2 + \\lvert\\lvert a - b \\rvert\\rvert^2 = 2(\\lvert\\lvert a \\rvert\\rvert^2 + \\lvert\\lvert b \\rvert\\rvert^2)$\\\\\
    $\\lvert\\lvert a + b \\rvert\\rvert^2 = (a + b)^T(a + b)$\\\\\
    $ = a^Ta + 2a^Tb + b^Tb$\\\\\
    $\\lvert\\lvert a - b \\rvert\\rvert^2 = (a - b)^T(a - b)$\\\\\
    $ = a^Ta - 2a^Tb + b^Tb$\\\\\
    $\\lvert\\lvert a + b \\rvert\\rvert^2 + \\lvert\\lvert a - b \\rvert\\rvert^2 = a^Ta + 2a^Tb + b^Tb + a^Ta - 2a^Tb + b^Tb$\\\\\
    $\\lvert\\lvert a + b \\rvert\\rvert^2 + \\lvert\\lvert a - b \\rvert\\rvert^2 = 2a^Ta + 2b^Tb$\\\\\
    Since $a^Ta = \\lvert\\lvert a \\rvert\\rvert^2$ and $b^Tb = \\lvert\\lvert b \\rvert\\rvert^2$\\\\\
    $\\lvert\\lvert a + b \\rvert\\rvert^2 + \\lvert\\lvert a - b \\rvert\\rvert^2 = 2\\lvert\\lvert a \\rvert\\rvert^2 + 2\\lvert\\lvert b \\rvert\\rvert^2$\\\\\
    $\\therefore \\lvert\\lvert a + b \\rvert\\rvert^2 + \\lvert\\lvert a - b \\rvert\\rvert^2 = 2(\\lvert\\lvert a \\rvert\\rvert^2 + \\lvert\\lvert b \\rvert\\rvert^2)$\
\\end\{enumerate\}\
\\end\{solution\}\
\
\\begin\{problem\}\{3.5\}\{6\}\
\\textit\{General norms\}. Any real-valued function $f$ that satisfies the four properties given on page \{\\color\{red\} 46\} (nonnegative homogeneity, triangle inequality, nonnegativity, and definiteness) is called a \\textit\{vector norm\}, and is usually written as $f(x) = \\lvert\\lvert x \\rvert\\rvert_\{mn\}$, where the subscript is some kind of identifier or mnemonic to identify it. The most commonly used norm is the one we use in this book, the Euclidean norm, which is sometimes written with the subscript 2, as $\\lvert\\lvert x \\rvert\\rvert_2$. Two other common vector norms for $n$-vectors are the 1-\\textit\{norm\}$\\lvert\\lvert x \\rvert\\rvert_1$ and the \\infty-\\textit\{norm\} $\\lvert\\lvert x \\rvert\\rvert_\{\\infty\}$, defined as $$\\lvert\\lvert x \\rvert\\rvert_1 = \\lvert x_1 \\rvert + \\cdots + \\lvert x_n \\rvert , \\quad \\lvert\\lvert x \\rvert\\rvert_\{\\infty\} = \\text\{max\}\\\{\\lvert x_1 \\rvert , \\ldots , \\lvert x_n \\rvert\\\}.$$ These Norms are the sum and the maximum of the of the absolute values of the entries in the vector, respectively. The 1-norm and the $\\infty$-norm arise in some recent and advanced applications, but we will not encounter them in this book. \\\\\
Verify the 1-norm and the $\\infty$-norm satisfy the four norm properties listed on page \{\\color\{red\} 46\}.\
\\end\{problem\}\
\\begin\{solution\}\
We will first prove \\textit\{Nonnegative homogenity\}: $\\lvert\\lvert\\beta x\\rvert\\rvert = \\lvert\\beta\\rvert\\lvert\\lvert x \\rvert\\rvert$.\
\\begin\{enumerate\}[(a)]\
    \\item $\\lvert\\lvert\\beta x\\rvert\\rvert_1 = \\lvert\\beta\\rvert\\lvert\\lvert x \\rvert\\rvert_1$\\\\\
    $\\lvert\\lvert\\beta x\\rvert\\rvert_1 = \\lvert\\beta x_1 \\rvert + \\lvert\\beta x_2 \\rvert + \\cdots + \\lvert\\beta x_n \\rvert$\\\\\
    $\\lvert\\lvert\\beta x\\rvert\\rvert_1 = \\lvert\\beta\\rvert(\\lvert x_1 \\rvert + \\lvert x_2 \\rvert + \\cdots + \\lvert x_n \\rvert)$\\\\\
    $\\lvert\\lvert x\\rvert\\rvert_1 = (\\lvert x_1 \\rvert + \\lvert x_2 \\rvert + \\cdots + \\lvert x_n \\rvert)$\\\\\
    $\\lvert\\lvert\\beta x\\rvert\\rvert_1 = \\lvert\\beta\\rvert\\lvert\\lvert x\\rvert\\rvert_1$\
    \\item $\\lvert\\lvert\\beta x\\rvert\\rvert_\\infty = \\lvert\\beta\\rvert\\lvert\\lvert x \\rvert\\rvert_\\infty$\\\\\
    $\\lvert\\lvert\\beta x\\rvert\\rvert_\\infty = \\max(\\lvert\\beta x_1 \\rvert, \\lvert\\beta x_2 \\rvert, \\ldots , \\lvert\\beta x_n \\rvert)$\\\\\
    $\\lvert\\lvert\\beta x\\rvert\\rvert_\\infty = \\lvert\\beta\\rvert\\max(\\lvert x_1 \\rvert,\\lvert x_2 \\rvert, \\ldots , \\lvert x_n \\rvert)$\\\\\
    $\\lvert\\lvert\\beta x\\rvert\\rvert_\\infty = \\lvert\\beta\\rvert\\lvert\\lvert x \\rvert\\rvert_\\infty$\
\\end\{enumerate\}\
Next we will prove \\textit\{triangle inequality\}: $\\lvert\\lvert x + y\\rvert\\rvert\\leq\\lvert\\lvert x\\rvert\\rvert + \\lvert\\lvert y\\rvert\\rvert$\\\\\
\\begin\{enumerate\}[(a)]\
    \\item $\\lvert\\lvert x + y\\rvert\\rvert_1 = \\lvert x_1 + y_1\\rvert + \\cdots + \\lvert x_n + y_n\\rvert$\\\\\
    $\\lvert\\lvert x \\rvert\\rvert_1 = \\lvert x_1\\rvert + \\cdots + \\lvert x_n\\rvert$\\\\\
    $\\lvert\\lvert y \\rvert\\rvert_1 = \\lvert y_1\\rvert + \\cdots + \\lvert y_n\\rvert$\\\\\
    If we let $x_i > 0$ and $y_i < 0$\
    $$\\lvert x_i + (-y_i)\\rvert \\leq \\lvert x_i\\rvert + \\lvert y_i\\rvert$$\
    $$\\lvert x_i - y_i\\rvert<\\lvert x_i\\rvert+\\lvert y_i\\rvert$$\
    If the signs of $x_i$ and $y_i$ are the same\
    $$\\lvert-x_i + (-y_i)\\rvert \\leq \\lvert x_i\\rvert + \\lvert y_i\\rvert$$\
    $$\\lvert x_i + y_i\\rvert = \\lvert x_i\\rvert + \\lvert y_i\\rvert$$\
    Therefore, for our 1-norm of the sum of two vectors, at each index we will either get a value less than $\\lvert x_i\\rvert$ + $\\lvert y_i\\rvert$, or one that is strictly less than that value. So when we sum, the greatest value we could get would be the sum of 1-norms of each vector, or less.\\\\\
    $$\\lvert\\lvert x + y\\rvert\\rvert_1\\leq\\lvert\\lvert x\\rvert\\rvert_1 + \\lvert\\lvert y\\rvert\\rvert_1$$\\\\\
    \\item $\\lvert\\lvert x + y\\rvert\\rvert_\\infty = \\max(\\lvert x_1 + y_1\\rvert, \\ldots, \\lvert x_n + y_n\\rvert)$\\\\\
    If we let $\\Vec\{c\} = x + y$, using the $\\infty$-norm we are going to get the value that is the greatest in the vector of sums, $\\Vec\{c\}$. However, with $\\infty$-norm on each individual vector $x$ and $y$ we can get two different indices that correlate to each of the respective vector's largest value.\\\\\
    $$\\lvert\\lvert x + y\\rvert\\rvert_\\infty = \\max(\\lvert x_1 + y_1\\rvert, \\ldots, \\lvert x_n + y_n\\rvert)$$\
    $$\\lvert\\lvert x + y\\rvert\\rvert_\\infty =\\lvert x_i + y_i\\rvert$$\
    $$\\lvert\\lvert x \\rvert\\rvert_\\infty = \\max(\\lvert x_1 \\rvert, \\ldots, \\lvert x_n \\rvert) \\quad\
    \\lvert\\lvert y \\rvert\\rvert_\\infty = \\max(\\lvert y_1 \\rvert, \\ldots, \\lvert y_n \\rvert)$$\
    $$\\lvert\\lvert x \\rvert\\rvert_\\infty = \\lvert x_i \\rvert \\quad \\lvert\\lvert y \\rvert\\rvert_\\infty = \\lvert y_i \\rvert$$\
    $$\\forall x_i, y_i \\in x, y$$\
    Let $x_i$ be positive and $y_i$ be negative.\
    $$\\lvert x_i +(-y_i)\\rvert\\leq\\lvert x_i\\rvert+\\lvert y_i\\rvert$$\
    $$\\lvert x_i - y_i\\rvert<\\lvert x_i\\rvert+\\lvert y_i\\rvert$$\
    Now, let $x_i$ and $y_i$ carry the same sign.\
    $$\\lvert -x_i +(-y_i)\\rvert\\leq\\lvert x_i\\rvert+\\lvert y_i\\rvert$$\
    $$\\lvert -x_i - y_i\\rvert\\leq\\lvert x_i\\rvert+\\lvert y_i\\rvert$$\
    $$x_i + y_i=\\lvert x_i\\rvert+\\lvert y_i\\rvert$$\
    So now we see that the triangle inequality holds because if it happens that the two largest absolute value elements are at equivalent indices, $i$, and have opposite sign, then it will be strictly less than the sum of the absolute value of $x_i$ and the absolute value of $y_i$. If both $i$th elements have the same sign, than the absolute value of their sum will be equal to the absolute value of $x_i$ and the absolute value of $y_i$.\
    $$\\lvert\\lvert x + y\\rvert\\rvert_\\infty\\leq\\lvert\\lvert x\\rvert\\rvert_\\infty + \\lvert\\lvert y\\rvert\\rvert_\\infty$$\
\\end\{enumerate\}\
\\newpage\
Now we will prove \\textit\{Nonnegativity\}. $\\lvert\\lvert x \\rvert\\rvert \\geq 0$\
\\begin\{enumerate\}[(a)]\
    \\item $\\lvert\\lvert x \\rvert\\rvert_1 \\geq 0$\\\\\
    $\\lvert\\lvert x \\rvert\\rvert_1 = (\\lvert x_1 \\rvert + \\lvert x_2 \\rvert + \\cdots + \\lvert x_n \\rvert)$\\\\\
    Even if x is made entirely of values such that $x_i < 0, \\forall x_i \\in x$, we can not get a negative value returned. The smallest value we could get would be 0, which would be an $n$-vector of zeroes. Since the sum of non-negative integers is non-negative, 1-norm can not be negative. \
    \\item $\\lvert\\lvert x \\rvert\\rvert_\\infty \\geq 0$\\\\\
    $\\lvert\\lvert x \\rvert\\rvert_\\infty = \\max(\\lvert x_1 \\rvert, \\lvert x_2 \\rvert, \\ldots, \\lvert x_n \\rvert)$\\\\\
    $\\lvert\\lvert x \\rvert\\rvert_\\infty = \\lvert x_i\\rvert$\\\\\
    Since we are taking the absolute value, if $x_i < 0$:\
    $$ -x_i < 0$$\
    $$\\lvert -x_i\\rvert < 0$$\
    $$+x_i > 0$$\
    So, $\\lvert\\lvert x \\rvert\\rvert_\\infty = \\max(\\lvert x_1 \\rvert, \\lvert x_2 \\rvert, \\ldots, \\lvert x_n \\rvert)$\\\\\
    $\\lvert\\lvert x \\rvert\\rvert_\\infty = \\lvert x_i \\rvert$\\\\\
    $\\lvert x_i \\rvert \\geq 0, \\lvert\\lvert x \\rvert\\rvert_\\infty \\geq 0$\
\\end\{enumerate\}\
Now we will prove \\textit\{Definiteness\}. $\\lvert\\lvert x \\rvert\\rvert = 0$, if and only if $x = 0$.\
\\begin\{enumerate\}[(a)]\
    \\item $\\lvert\\lvert x \\rvert\\rvert_1 = 0$\\\\\
    $(\\lvert x_1 \\rvert + \\lvert x_2 \\rvert + \\cdots + \\lvert x_n \\rvert) = 0$\\\\\
    $0 + 0 + \\cdots + 0 = 0$\\\\\
    The only way to sum a vector of absolute values to zero, is to sum all zeros.\
    \\item $\\lvert\\lvert x \\rvert\\rvert_\\infty = 0$\\\\\
    $\\max(\\lvert x_1 \\rvert,\\lvert x_2 \\rvert, \\ldots, \\lvert x_n \\rvert) = 0$\\\\\
    $\\max(\\lvert 0 \\rvert,\\lvert 0 \\rvert, \\ldots, \\lvert 0 \\rvert) = 0$\\\\\
    $0 = 0$\
\\end\{enumerate\}\
\\end\{solution\}\
\\newpage\
\\begin\{problem\}\{3.7\}\{2\}\
\\textit\{Chebyshev inequality\}. Suppose $x$ is a 100-vector with \\textbf\{rms\}$(x) = 1$. What is the maximum number of entries of $x$ that can satisfy $\\lvert x_i \\rvert \\geq 3$? If your answer is $k$, explain why no such vector can have $k+1$ entries with absolute values of at least 3, give an example of a specific 100-vector that has RMS value 1, with $k$ of its entries larger than 3 in absolute value.\
\\end\{problem\}\
\\begin\{solution\}\
\\begin\{align*\}\
    \\frac\{k\}\{n\} &\\leq \\bigg( \\frac\{\\textbf\{rms\}(x)\}\{a\}\\bigg)^2\\\\\
    k &\\leq n\\bigg(\\frac\{\\textbf\{rms\}(x)\}\{a\}\\bigg)^2\\\\\
\\end\{align*\}\
Here we have $n=100$, $a=3$, and $\\textbf\{rms\}(x)=1$.\
\\begin\{align*\}\
    k &\\leq 100 \\bigg(\\frac\{1\}\{3\}\\bigg)^2\\\\\
    k &\\leq 100 \\bigg(\\frac\{1\}\{9\}\\bigg)\\\\\
    k &\\leq \\frac\{100\}\{9\}\\\\\
    k &\\leq 11.1\\\\\
\\end\{align*\}\
$$\\boxed\{k = 11\}$$\
The reason we have $k = 11$ and not 11.1 is because it's an integer value, it's the number of entries that are above that value. If $k=12$ we would violate the inequality. \
$$\\frac\{12\}\{100\}\\leq\\frac\{1\}\{9\}$$\
$$.12\\nleq .\\overline\{11\}$$\
\\end\{solution\}\
\
\\begin\{problem\}\{3.8\}\{4\}\
\\textit\{Converse Chebyshev inequality\}. Show that at least one entry of a vector has absolute value at least as large as the RMS value of the vector.\
\\end\{problem\}\
\\begin\{solution\}\
Since a mean is an average we will take $x_1$ to be the highest value in an $n$-vector.\
\\begin\{align*\}\
    x_1 &\\leq \\textbf\{RMS\}(x)\\\\\
    x_1 &\\leq \\sqrt\{\\frac\{x_1^2 + x_2^2 + \\cdots + x_n^2\}\{n\}\}\\\\\
    (x_1)^2 &\\leq \\bigg(\\sqrt\{\\frac\{x_1^2 + x_2^2 + \\cdots + x_n^2\}\{n\}\}\\bigg)^2\\\\\
    x_1^2 &\\leq \\frac\{x_1^2 + x_2^2 + \\cdots + x_n^2\}\{n\}\\\\\
    nx_1^2 &\\leq x_1^2 + x_2^2 + \\cdots + x_n^2\
\\end\{align*\}\
Now since we are multiplying $x_1^2$ by a factor of $n$, we can rewrite it as $x_1^2$ added to itself n times which leaves\
$$x_1_1^2 + x_1_2^2 + \\cdots + x_1_n^2 \\leq x_1^2 + x_2^2 + \\cdots + x_n^2$$\
and since we said $x_1$ was the max value of $x$, it's squared sum mutiplied by $n$ would be larger than the sum of squares of $n$ elements in $x$, which is not what our inequality says. Thus, there must be at least one entry in $x$ such that $\\boxed\{\\lvert x_i\\rvert \\geq \\textbf\{RMS\}(x)\}$, unless $x = 0$.\
\\end\{solution\}\
\\newpage\
\\begin\{problem\}\{3.22\}\{4\}\
\\textit\{Distance from Palo Alto to Beijing\}. The surface of the earth is reasonably approximated as a sphere with radius $R = 6367.5$km. A location on the Earth's surface is traditionally given by its latitude $\\theta$ and its longitude $\\lambda$, which correspond to angular distance from the equator and prime meridian, respectively. The 3-D coordinates of the location are given by\
$$\\begin\{bmatrix\} R\\sin\{\\lambda\}\\cos\{\\theta\}\\\\ R\\cos\{\\lambda\}\\cos\{\\theta\} \\\\ R\\sin\{\\theta\} \\end\{bmatrix\}.$$\
(In this coordinate system $(0,0,0)$ is the center of the earth, $R(0,0,1)$ is the North pole, and $R(0,1,0)$ is the point on the equator on the prime meridian, due south of the Royal Observatory outside London.)\\\\\
The distance \\textit\{through the earth\} between two locations (3-vectors) $a$ and $b$ is $\\lvert\\lvert a - b \\rvert\\rvert$. The distance \\textit\{along the surface of the earth\} between two points $a$ and $b$ is $R\\angle(a,b)$. Find these two distances between Palo Alto and Beijing, with Latitudes and longitudes given below.\\\\\
\\begin\{center\}\
 \\begin\{tabular\}\{c c c\} \
 \\hline\
 City & Latitude $\\theta$ & Longitude $\\lambda$ \\\\ [0.5ex] \
 \\hline\\hline\
 Beijing & 39.914\\degree & 116.392\\degree \\\\\
 Palo Alto & 37.429\\degree & -122.138\\degree \\\\[1ex] \
 \\hline\
\\end\{tabular\}\
\\end\{center\}\
\\end\{problem\}\
\\begin\{solution\}\
We will begin with $a - b$\
\\begin\{align*\}\
    a-b &= \\begin\{bmatrix\} a_1\\\\ a_2 \\\\ a_3 \\end\{bmatrix\} - \\begin\{bmatrix\} b_1\\\\ b_2 \\\\ b_3 \\end\{bmatrix\}\\\\\
     &= \\begin\{bmatrix\} R\\sin\{\\lambda_\{PA\}\}\\cos\{\\theta_\{PA\}\}\\\\ R\\cos\{\\lambda_\{PA\}\}\\cos\{\\theta_\{PA\}\} \\\\ R\\sin\{\\theta_\{PA\}\} \\end\{bmatrix\} - \\begin\{bmatrix\} R\\sin\{\\lambda_\{B\}\}\\cos\{\\theta_\{B\}\}\\\\ R\\cos\{\\lambda_\{B\}\}\\cos\{\\theta_\{B\}\} \\\\ R\\sin\{\\theta_\{B\}\} \\end\{bmatrix\}\\\\\
     &= \\begin\{bmatrix\} 6367.5\\sin(-122.138)\\cos(37.429)\\\\ 6367.5\\cos(-122.138)\\cos(37.429) \\\\ 6367.5\\sin(37.429) \\end\{bmatrix\} - \\begin\{bmatrix\} 6367.5\\sin(116.392)\\cos(39.914)\\\\ 6367.5\\cos(116.392)\\cos(39.914) \\\\ 6367.5\\sin(39.914) \\end\{bmatrix\}\\\\\
     &= \\begin\{bmatrix\} -4281.670 \\\\ -2689.845 \\\\ 3870.025 \\end\{bmatrix\} - \\begin\{bmatrix\} 4374.893 \\\\ -2170.954 \\\\ 4085.624 \\end\{bmatrix\}\\\\\
     &= \\begin\{bmatrix\} -8656.563 \\\\ -518.891 \\\\ -215.599 \\end\{bmatrix\}\
\\end\{align*\}\
Now we will find the Euclidean Norm.\
$$\\sqrt\{(-8656.563)^2 + (-518.891)^2 + (-215.624)^2\}$$\
Which returns $\\boxed\{8674.78\}$ km through the earth from each other.\\\\\
\\\\ Now we will find the distance along the surface, $R\\angle(a,b)$. First we need to find $\\angle(a,b)$\
$$\\angle(a,b) = \\arccos\\bigg(\\frac\{a^T b\}\{\\lvert\\lvert a \\rvert\\rvert \\lvert\\lvert b \\rvert\\rvert\}\\bigg)$$\
First we can find $a^Tb$\
\\begin\{align*\}\
    a^Tb &= a_1b_1 + a_2b_2 +a_3b_3\\\\\
    a^Tb &= -18731848.11 + 583952.762 + 15811467.02\\\\\
    a^Tb &= 2919148.672\
\\end\{align*\}\
Then we will find both $\\lvert\\lvert a \\rvert\\rvert$ and $\\lvert\\lvert b \\rvert\\rvert$\
\\begin\{align*\}\
    \\lvert\\lvert a \\rvert\\rvert &= \\sqrt\{(-4281.670)^2 + (-2689.845)^2 + (3870.025)^2\} & \\lvert\\lvert b \\rvert\\rvert &= \\sqrt\{(4374.893)^2 + (-2170.954)^2 + (4085.624)^2\}\\\\\
    \\lvert\\lvert a \\rvert\\rvert &= 6367.5 & \\lvert\\lvert b \\rvert\\rvert &= 6367.499\
\\end\{align*\}\
So now we plug and chug:\
$$R\\arccos\\bigg(\\frac\{2919148.672\}\{(6367.5)(6367.499)\}\\bigg)$$\
$$(6367.5)\\arccos(.07199766)$$\
$\\arccos\{(.0719976)\} = 85.871\\degree$, but we need radians:\
$$(6367.5)\\frac\{(85.8712)(\\pi)\}\{180\}$$\
$$(6367.5)(1.4987)$$\
And we get $\\boxed\{9543.2\}$ km along the surface.\
\\end\{solution\}\
\\end\{document\}}